% Informationen ------------------------------------------------------------
% 	Definition von globalen Parametern, die im gesamten Dokument verwendet
% 	werden können (z.B auf dem Deckblatt etc.).
% --------------------------------------------------------------------------
\newcommand{\titel}{\"Ubersetzen von Schrittmotorprotokollen}
\newcommand{\untertitel}{Entwurf eines Hardware\"ubersetzers}
\newcommand{\art}{Praxisbericht}
\newcommand{\fachgebiet}{Mess- und Sensortechnik}
\newcommand{\autor}{Johannes Dielmann}
\newcommand{\studienbereich}{Technik}
\newcommand{\matrikelnr}{515956}
\newcommand{\erstgutachter}{Prof. Dr. Carstens-Behrens}
\newcommand{\zweitgutachter}{Prof. Dr. }
\newcommand{\jahr}{2012}

% Eigene Befehle und typographische Auszeichnungen für diese
\newcommand{\todo}[1]{\textbf{\textsc{\textcolor{red}{(TODO: #1)}}}}

\newcommand{\AutorZ}[1]{\textsc{#1}}
\newcommand{\Autor}[1]{\AutorZ{\citeauthor{#1}}}

\newcommand{\NeuerBegriff}[1]{\textbf{#1}}

\newcommand{\Fachbegriff}[1]{\colorbox{hellblau}{#1}}
\newcommand{\Prozess}[1]{\textit{#1}}
\newcommand{\Webservice}[1]{\textit{#1}}

\newcommand{\Eingabe}[1]{\texttt{#1}}
\newcommand{\Code}[1]{\texttt{\colorbox{hellgelb}{\textcolor{colIdentifier}{#1}}}}
\newcommand{\Datei}[1]{\texttt{\colorbox{hellgelb}{#1}}}

\newcommand{\Datentyp}[1]{\textsf{#1}}
\newcommand{\XMLElement}[1]{\textsf{#1}}


% Abkürzungen mit korrektem Leerraum
\newcommand{\vgl}{Vgl.\ }
\newcommand{\ua}{\mbox{u.\,a.\ }}
\newcommand{\zB}{\mbox{z.\,B.\ }}
\newcommand{\bs}{$\backslash$}

