\chapter{Vorstellung der vorhandenen Software}
\label{cha:Software}

\section{RapidForm2004}
\label{sec:RapidForm}
Zur Erfassung von 3D-Modellen am PC steht die Software RapidForm2004 der Firma INUS Technology Inc. zur Verfügung. Diese ist zur Erfassung und Bearbeitung von 3D-Modellen gedacht. Sie bietet umfangreiche Möglichkeiten die aufgenommenen Modelle zu verbessern, zu verändern, zu vermessen und in verschiedene Formate zu exportieren.\\
Die Ansteuerung des VI-900 ist durch ein \Fachbegriff{Add-In} bereits in die Software integriert. Das Add-In kann den VI-900 ansteuern und die aufgenommenen Daten auslesen. Weiterhin kann das Add-In verschiedene Drehtische ansteuern.

\section{Entwicklungsumgebung}
\label{sec:Entwicklungsumgebung}
Die von Atmel bereitgestellte Entwicklungsumgebung besteht aus einem Editor, dem Compiler und einer Programmiersoftware. Der Editor bietet Komfortfunktionen wie \Fachbegriff{Syntaxhighlighting}, Autovervollständigung und Projektmanagement.

\section{Terminalprogramme}
\label{sec:Terminal}
Als Terminalprogramm zur Kommunikation zwischen Datengeräten über die serielle Schnittstelle steht das Programm "Hypterminal" der Firma Microsoft zur Verfügung.


%\section{Weiteres}Pins am Mikrcontroller werden wie folgt angegeben PortX(a:b).Das X steht dabei für den Portbuchstaben von A--D. Die Buchstaben a und b stehen für die Pinnummer.\\PortB(3:6) würde also die Pins 3--6 an PortB beschreiben.