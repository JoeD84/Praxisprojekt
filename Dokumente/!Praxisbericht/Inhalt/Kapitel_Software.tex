\chapter{Vorstellung der vorhandenen Software}
\label{cha:Software}

\section{RapidForm2004}
\label{sec:RapidForm}
Zur Erfassung von 3D-Modellen am PC steht die Software RapidForm2004 der Firma INUS Technology Inc. zur Verfügung. Diese ist zur Erfassung und Bearbeitung von 3D-Modellen gedacht. Sie bietet umfangreiche Möglichkeiten die aufgenommenen Modelle zu verbessern, zu verändern, zu vermessen und in verschiedene Formate zu exportieren.\\
Die Ansteuerung des VI-900 ist durch ein \Fachbegriff{Add-In} bereits in die Software integriert. Das Add-In kann den VI-900 ansteuern und die aufgenommenen Daten auslesen. Weiterhin kann das Add-In verschiedene Drehtische ansteuern.

\section{Entwicklungsumgebung}
\label{sec:Entwicklungsumgebung}
Die von Atmel bereitgestellte Entwicklungsumgebung besteht aus einem Editor, dem Compiler und einer Programmiersoftware. Der Editor bietet Komfortfunktionen wie \Fachbegriff{Syntaxhighlighting}, Autovervollständigung und Projektmanagement.

\section{Terminalprogramme}
\label{sec:Terminal}
Als Terminalprogramm zur Kommunikation zwischen Datengeräten über die serielle Schnittstelle steht das Programm "Hypterminal" der Firma Microsoft zur Verfügung.

%\subsection{AVR Studio 5}
%Die von Atmel AVR Studio 5 ist eine von Atmel bereitgestellte Entwicklungsumgebung. Diese scheint jedoch eine fehlerhafte Bibliothek zu enthalten. Die Kombination aus Mikrocontroller ATmega324A und AVR Studio 5 erzeugte nicht nachvollziehbare Probleme. Bei dem selbem Programm und einem anderem Mikrocontroller oder einer anderen Entwicklungsumgebung tauchten keine Fehler auf.
%In der Entwicklungsumgebung Eclipse lies sich der Fehler reproduzieren wenn der Pfad der Atmel Bibliotheken eingestellt wurde. Die WinAVR Bibliotheken und eine selbst kompilierte \Fachbegriff{Toolchain} unter Linux zeigten keine Probleme.\\
%Daher wechselte ich zur \Fachbegriff{Open Source} Entwicklungsumgebung Eclipse. Erst dadurch wurde es möglich erfolgreich zu arbeiten. Außerdem wurde das Projekt dadurch plattformunabhänig und ich nutzte bis auf RapidForm2004 nur noch freie Open Source Software.\\
%\subsection{Eclipse}
%Eclipse ist eine in Java programmierte freie Open Source Entwicklungsumgebung für Java. Sie lässt sich durch \Fachbegriff{Plugins} leicht für viele Sprachen erweitern.\\
%Mit dem CDT-Plugin, dem AVR-Plugin und einer Bibliothek wie z.B. WinAVR für Windows ist Eclipse eine vollwertige Entwicklungsumgebung für Atmel Mikrocontroller. 
%Ergänzt wird diese durch die Programmiersoftware AVR-Dude.\\












