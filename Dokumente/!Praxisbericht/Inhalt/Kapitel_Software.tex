\chapter{Vorstellung der vorhandenen Software}
\label{cha:Software}

\section{RapidForm2004}
\label{sec:RapidForm}
Zur Erfassung von 3D-Modellen am PC steht die Software \emph{RapidForm2004}[\ref{sec:V_Software}] zur Verfügung. Diese ist zur Erfassung und Bearbeitung von 3D-Modellen gedacht. Sie bietet umfangreiche Möglichkeiten die aufgenommenen Modelle zu verbessern, zu verändern, zu vermessen und in verschiedene Formate zu exportieren.\\
Mittels eines \Fachbegriff{Add-In} kann der VI-900 angesteuert und aufgenommenen Daten ausgelesen werden. Weiterhin kann das Add-In über eine RS-232-Schnittstelle verschiedene Drehtische ansteuern.

\section{Entwicklungsumgebung}
\label{sec:Entwicklungsumgebung}
Die von Atmel bereitgestellte Entwicklungsumgebung \emph{AVR Studio 5}[\ref{sec:V_Software}] besteht aus einem Editor, einem Compiler und einer Programmiersoftware. Der Editor bietet Komfortfunktionen wie \Fachbegriff{Syntaxhighlighting}, Autovervollständigung und Projektmanagement. Der Compiler übersetzt den Quelltext in einen maschinenlesbaren Code und die Programmiersoftware kann diesen auf einen Mikrocontroller spielen.

\section{Terminalprogramme}
\label{sec:Terminal}
Zur Kommunikation über die RS-232-Schnittstelle steht das Programm \emph{Hypterminal}[\ref{sec:V_Software}] zur Verfügung. Dieses wurde im Verlauf des Projekts durch das wesentlich umfangreichere Open Source Programm \emph{Putty}[\ref{sec:V_Software}] abgelöst.