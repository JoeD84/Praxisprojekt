\chapter{Einleitung}
\label{cha:Einleitung}
\todo{Die Einleitung ist sehr WICHTIG!!!}
\todo{Klarstellen der Begrifflichkeiten}
In der \todo{CAD erklären} \Fachbegriff{CAD-Entwicklung} kommt es vor das für ein real existierendes Objekt eine Erweiterung konstruiert werden muss. Um die Erweiterung sinnvoll konstruieren zu können müssen dazu die Abmessungen des Objektes möglichst genau bekannt sein. Das übertragen der Abmessungen, kann insbesondere für komplexe Objekte, sehr aufwendig sein.
Abhilfe soll ein Laserscanner schaffen der das Objekt aus mehreren Richtungen vermisst und aus diesen Informationen ein genaues 3D-Modell davon generiert. 

\section{Motivation}
\label{sec:Motivation}
Mit dem Aufbau aus RapidForm2004, Lasererfassungssystem VI-900 und Drehtisch sollen auf einfachem Wege 3D-Modelle eines Objektes erzeugt werden. Diese sollen anschließend in einer CAD-Software wie \Fachbegriff{Solidworks} nutzbar sein.

\section{Ziel der Arbeit}
\label{sec:ZielDerArbeit}
Die Kommunikation zwischen der Software RapidForm2004 und dem gegebenen Drehtisch soll ermöglicht werden. Dazu werden die ASCII-Befehle der Software mit einem Mikrocontroller ausgewertet und in für den gegeben Schrittmotor verständliche Befehle übersetzt.
Es wird also ein Mikrocontroller mit 2 RS-232 Schnittstellen so programmiert das er die Befehle der Software übersetzen kann. Um den den Drehtisch manuell bedienen zu können und den aktuellen Status zu überprüfen sind noch ein LC-Display und mehrere Bedientaster vorgesehen.\\
Die Ansteuerung des Schrittmotors ist als Einschub für ein 19"-Rack realisiert. Daher wird die Platine für den Mikrocontroller auch als 19"-Einschub realisiert.


\section{Aufbau der Arbeit}
\label{sec:AufbauDerArbeit}
Der Aufbau der Arbeit gliedert sich im Wesentlichen in die Entwicklung neuer und die Nutzung vorhandener Hardware, sowie in die Entwicklung der Software. \\
Zur Hardware gehören die Auswahl des Mikrocontroller, die Endschalter, die Schrittmotorkarten, die Schrittmotoren, sowie die verwedeten PCs.
\todo{Kabel zu den Motoren nicht vergessen! Schema Zeichnung?}
Zur Software gehören die Entwicklungsumgebungen und die 3D-Erfassungssoftware.

\todo{Ausbauen! Überblick verschaffen. Kommunikation mit Schrittmotor-Karte von PC aus. Aufbauen des STK500. Einarbeiten in uC Entwicklung. Steuern der Schrittmotor Karte vom uC aus. Erarbeiten der Protokolle (Reverse Engineering). C-Programm zum verstehen eingehender Befehle. Übersetzen der Befehle. Neuer Mikrocontroller. Umgebungs wechsel. LC-Display. Endschalter. Max232. Platinenlayout.}