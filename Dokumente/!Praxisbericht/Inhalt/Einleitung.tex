\chapter{Einleitung}
\label{cha:Einleitung}
Ein 3D-Laserscanner bietet vielfältige Möglichkeiten und Einsatzgebiete. Die Haupt-einsatzgebiete finden sich in der Bauteileprüfung, zur Erstellung von Finite-Elemente-Daten in Verbindung mit Bauteilanalyse, zur Erstellung von 3D-Daten, zur Kontrolle von Zubehörteilen, zur Archivierung in Museen und Forschungseinrichtungen, zum Reverse-Engineering und in vielen weiteren Gebieten.\\
Im Besitz der Fachhochschule befand sich ein komplettes 3D-Lasererfassungssystem. Dazu gehörten eine Erfassungssoftware, ein 3D-Laserscanner und ein Drehtisch. Bisher mussten, um eine Aufnahme zu tätigen, alle Komponenten zueinander passen. Der Drehtisch in diesem System war jedoch ein Eigenbau der Fachhochschule und die darin verbaute Drehtischsteuerung nicht kompatibel zu denen von der Erfassungssoftware unterstützten Drehtischsteuerungen.\\
Mittels eines Mikrocontrollers sollte der vorhandene Aufbau so erweitert werden, dass der Drehtisch von der Software angesteuert werden konnte und so der volle Umfang des Systems nutzbar gemacht werden. Dabei waren folgende Vorgaben zu realisieren. Die Höhenverstellung des Drehtisches sollte genutzt werden können und die verbauten Endschalter ihre vorhergesehene Funktion erfüllen. 
Der Mikrocontroller sollte sich mit mehreren Tastern bedienen lassen und über ein LC-Display verfügen, welches den aktuellen Status anzeigt. Mit einer Schritt-für-Schritt-Anleitung sollte es auch für Studenten und Mitarbeiter der Fachhochschule möglich sein, schnell und einfach eine Aufnahme durchzuführen. Die Daten dieser Aufnahme sollten exportiert und in z.B. CAD-Software genutzt werden können. \\
Der Aufbau der Arbeit gliedert sich im Wesentlichen in die Vorstellung der Vorhanden Hard- und Software, den chronologischen Arbeitsablauf während des Projektes, ein Kapitel das Probleme und deren Lösungen aufzeigt, in ein Fazit und mögliche zukünftige Verbesserungen. Im Anhang befindet sich eine Schritt-für-Schritt-Anleitung die es Laien ermöglicht 3D-Modelle aufzunehmen und zu exportieren.
