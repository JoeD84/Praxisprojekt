\chapter{Hardware}
\label{sec:Hardware}
\section{Mikrocontroller}
Zu Beginn stand mir ein ATmega 8515\cite{atmel_8515} im DIL-Gehäuse zur Verfügung. Dieser hatte 8Kbyte Flash, 3 externe Interrupts, 1 Serielle Schnittstelle und konnte mit bis zu 16 MHz betrieben werden. Mit diesem konnte ich mich in die Programmierung mit C einfinden und eine Serielle Schnittstelle ansteuern. Für mein Projekt sind jedoch 2 externe Schnittstellen nötig. Nach Recherche entschloss ich mich für einen ATmega 644 PA. \todo{verlinken} Dieser ist dem ATmega 8515 recht ähnlich, bietet jedoch die benötigten 2 seriellen Schnittstellen. Des weiteren hat er 32Kbyte Flash und verfügt über 32 externe Interrupts. \todo{mehr schreiben??}
\subsection{MAX232}
\section{Schrittmotorkarten}
\subsection{Endschalter}
\todo{Verkabelung, neuer Auslöser, Entwicklung der Verdrahtung, Berücksichtigung im uC}
\subsection{Motorverkabelung}
\section{Spannungsversorgung}
\todo{Verkabelung Steckbar und universell gemacht}
\section{Platinenlayout}