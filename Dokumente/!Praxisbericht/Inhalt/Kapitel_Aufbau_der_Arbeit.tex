\chapter{Aufbau der Arbeit}
\label{sec:AufbauDerArbeit}

\section{Erste Schritte}
Im ersten Schritt ging es darum, den Drehtisch mithilfe des Mikrocontrollers um 90° zu drehen. Der Mikrocontroller befindet sich auf dem STK 500(siehe Kapitel \ref{sec:STK500}). Dieses bietet grundlegenden Funktionalitäten wie Taster, LEDs, eine Programmierschnittstelle und eine serielle Schnittstelle.
Um die Komponenten sinnvoll im Mikrocontroller nutzen zu können müssen dafür Funktionalitäten wie z.B. Bibliotheken bereit gestellt werden oder Register initialisiert werden.\\
Die folgenden Kapitel beschreiben dieses bereitstellen der Funktionalitäten.
\subsection{Taster entprellen}
\label{sec:Taster}
Um Taster vernünftig nutzen zu können müssen diese Entprellt werden. Dazu kann die Bibliothek von Peter Fleury genutzt werden. Mit Zeile 1 des Code Listing 3.2 wird diese Bibliothek eingebunden. Zeilen 3-10 inintialisieren die Bibliothek.
Danach kann mit den Funktionen get\_key\_press der Status der Taster prellfrei ausgelesen werden.
\lstset{language=C, basicstyle=\footnotesize, showstringspaces=false, tabsize=2}
\lstinputlisting[label=lst:Taster,caption=Taster]{Code/taster.txt}

\subsection{LEDs ansteuern}
\subsection{LCD ansteuern}
\subsection{Serielle Schnittstelle ansteuern}	