\chapter{Netzwerkverkehr beim Aufruf von \Webservice{PersonFactory}}
\label{cha:SOAPNachrichten}

\section{SOAP-Request}
Listing \ref{lst:SOAPRequest} zeigt die mitgeschnittene SOAP-Anfrage per HTTP an den Webservice \Webservice{PersonFactory}. Wie am Ende von Kapitel \ref{cha:Einleitung} beschrieben, wird die eigentliche SOAP-Nachricht mittels des HTTP-\Eingabe{POST}-Befehls (Zeile 1) an den Webservice unter der angegebenen URL (Zeile 1) auf dem Server (Zeile 5) geschickt. In Zeile 3 wird über den Befehl \Eingabe{SOAPAction} übermittelt, welche Funktion des Webservice (in diesem Fall \Code{CreatePerson}) aufgerufen werden soll. Die XML-Nutzlast (Zeilen 8--18) besteht dann aus einer einfachen SOAP-Nachricht aus \XMLElement{Envelope}, \XMLElement{Header} und \XMLElement{Body}, die einen RPC durchführt. Die aufzurufenden Funktion wird noch einmal im SOAP-\XMLElement{Body} in Zeile 15 definiert.


\lstset{language=XML, basicstyle=\footnotesize, showstringspaces=false, tabsize=2}
\lstinputlisting[label=lst:SOAPRequest,caption=SOAP-Request an \Webservice{PersonFactory} per HTTP]{DVD/Listings/PersonFactorySOAPRequest.txt}

\section{SOAP-Response}
Die Antwort des \Webservice{PersonFactory}-Webservice zeigt Listing \ref{lst:SOAPResponse}. Sie beginnt in Zeile 1 mit dem HTTP-Statuscode 200, der die Anfrage als erfolgreich kennzeichnet. Die eigentliche Nutzlast in Form von XML-Daten (Zeile 3) folgt dann ab Zeile 7. Sie besteht aus dem Element \XMLElement{Person} und seinen Unterelementen, umschlossen vom Element \XMLElement{CreatePersonRepsonse}, das die Antwort-Nachricht aus der WSDL repräsentiert.

\lstset{language=XML, basicstyle=\footnotesize, showstringspaces=false, tabsize=2}
\lstinputlisting[label=lst:SOAPResponse,caption=SOAP-Response von \Webservice{PersonFactory} per HTTP]{DVD/Listings/PersonFactorySOAPResponse.txt}
