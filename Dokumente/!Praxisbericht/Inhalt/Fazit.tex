\chapter{Fazit und weitere Möglichkeiten}
\label{cha:Fazit}
\section{Fazit}
Das vorgegeben Ziel den Aufbau in Betrieb zu nehmen und mit einem Mikrocontroller so zu erweitern, dass die Erfassungssoftware RapidForm2004 kompatibel mit dem vorhandenen Drehtisch ist, wurde erreicht. Die Software kann vollständig genutzt werden und alle wesentlichen Funktionen der Schrittmotorsteuerungen werden von der Mikrocontroller Programmierung unterstützt. Im Anhang \ref{sec:StepbyStep} befindet sich eine Schritt-für-Schritt Anleitung mit der auch Laien das System nutzen können.
\section{Bekannte Probleme}
Bei einem abschließenden Test funktionierte das Anzeigen einer Meldung beim erreichen der Endschalter, auf dem Display nicht. Alle Verbindungen sind vorhanden und die Programmierung des Mikrocontrollers vollständig. Das Problem ist nicht bekannt und das Auffinden würde weitere Zeit in Anspruch nehmen.\\
Das Display zeigt während der Rotation \emph{0} anstatt dem Winkel an, um den rotiert wird. Für die Anzahl der Schritte funktionierte diese Anzeige. Vermutlich liegt hier ein Fehler in der Berechnung des Winkels vor.
\section{Weitere Möglichkeiten}
Eine elegantere Lösung als die Befehle der Befehlssätze in einem Array zu speichern wäre es diese in verketteten Pointer Structs zu speichern. Diese Idee kam leider erst gegen Ende des Projektes und konnte daher aus Zeitmangel nicht mehr umgesetzt werden.\\
Im Menü lassen sich zur Zeit nur voreingestellte Winkel bzw. Schrittzahlen auswählen. Hier könnte noch eine Funktion geschrieben werden die es dem Benutzer erlaubt Winkel frei einzustellen.
