\section{Technische Daten VI-910}
Die Technischen Daten beziehen sich auf den VI-910. Dies ist das Nachfolgemodell. Die meisten Daten sollten jedoch ähnlich sein.
\begin{longtable}{|m{4cm}|m{9cm}|} 
\caption{Technische Daten - VI-910}\\ \hline
\label{tab:TD_VI-910}
Modellbezeichnung & Optischer 3D-Scanner VI-910
 \\ \hline 
Messverfahren & Triangulation durch Lichtschnittverfahren
 \\ \hline 
Autofokus & Autofokus auf Objektoberfläche (Kontrastverfahren);  aktiver AF
 \\ \hline 
Objektive \newline  (wechselbar) & TELE Brennweite f=25mm  \newline 
MITTEL: Brennweite f=14 mm  \newline 
WEIT: Brennweite f=8mm
 \\ \hline 
Messabstand & 0,6 bis 2,5m (2m für WIDE-Objektiv)
 \\ \hline 
Optimaler Messabstand	 & 0,6 bis 1,2m
 \\ \hline 
Laserklasse & Class 2 (IEC60825-1), Class 1 (FDA)
 \\ \hline 
Laser-Scanverfahren	 & Galvanisch-angetriebener Drehspiegel
 \\ \hline 
Messbereich in  \newline X-Richtung (anhängig vom Anstand) & 111 bis 463mm (TELE),  \newline 198 bis 823mm (MITTEL), \newline  359 bis 1.196mm (WEIT)
 \\ \hline 
Messbereich in Y-Richtung (abhängig vom Abstand) & 83 bis 347mm (TELE), \newline  148 bis 618mm (MITTEL),  \newline 269 bis 897mm (WEIT)
 \\ \hline 
Messbereich in Z-Richtung (abhängig vom Abstand) & 40 bis 500mm (TELE), \newline  70 bis 800mm (MITTEL),  \newline 110 bis 750mm (WEIT/Modus FINE)
 \\ \hline 
Genauigkeit	 & X: ±0,22mm, Y: ±0,16mm, Z: ±0,10mm zur Z-Referenzebene (Bedingungen: TELE/Modus FINE , Konica Minolta Standard)
 \\ \hline 
Aufnahmezeit & 0,3s (Modus FAST), 2,5s (Modus FINE), 0,5s (COLOR)
 \\ \hline 
Übertragungszeit zum Host-Computer	 & ca. 1s (Modus FAST) oder 1,5s (Modus FINE)
 \\ \hline 
Scanumgebung, Beleuchtungsbedingungen	 & 500 lx oder geringer
 \\ \hline 
Aufnahmeeinheit & 3D-Daten: 1/3" CCD-Bildsensor (340.000 Pixel) 
Farbdaten: Zusammen mit 3D-Daten (Farbtrennung durch Drehfilter)
 \\ \hline 
Anzahl aufgenommener Punkte	 & 3D-Daten: 307.000 (Modus FINE), 76.800 (Modus FAST) 
Farbdaten: 640 × 480 × 24 Bit Farbtiefe
 \\ \hline 
Ausgabeformat & 3D-Daten: Konica Minolta Format, (STL, DXF, OBJ, ASCII-Punkte, VRML; Konvertierung in 3D-Daten durch Polygon Editing-Software / Standardzubehör)
Farbdaten: RGB 24-Bit Rasterscan-Daten
 \\ \hline 
Speichermedium & Compact Flash Memory Card (128MB)
 \\ \hline 
Dateigrößen & 3D- und Farbdaten (kombiniert): 1,6MB (Modus FAST) pro Datensatz, 3,6MB (Modus FINE) pro Datensatz 
 \\ \hline 
Monitor & 5,7"-LCD (320 × 240 Pixel)
 \\ \hline 
Datenschnittstelle & SCSI II (DMA-Synchronübertragung)
 \\ \hline 
Stromversorgung & Normale Wechselstromversorgung, 100V bis 240 V (50 oder 60 Hz), Nennstrom 0,6 A (bei 100 V) \\ \hline 
Abmessungen (B x H x T)	 & 213 × 413 × 271mm 
 \\ \hline 
Gewicht & ca. 11kg
 \\ \hline 
Zulässige Umgebungsbedingungen (Betrieb) & 10 bis 40°C; relative Luftfeuchtigkeit 65\% oder niedriger (keine Kondensation)
 \\ \hline 
Zulässige Umgebungsbedingungen (Lagerung) & -10 bis 50°C, relative Luftfeuchtigkeit 85\% oder niedriger (bei 35°C, keine Kondensation) 
 \\ \hline 
\end{longtable} 