\section{Schritt für Schritt Anleitung}
\label{sec:StepbyStep}
Eine Schritt für Schritt Anleitung zum vollständigen Scannen und exportieren eines 3D-Objektes.

\begin{longtable}{|>{\RaggedRight}m{5cm}|m{8cm}|} 
\caption{Schritt für Schritt Anleitung} 
\label{tab:StepbyStep}
\\ \hline
\multicolumn{2}{|c|}{\textbf{Schritt für Schritt Anleitung}}
\\ \hline 
\endfirsthead


\multicolumn{2}{|c|}%
{{ Fortsetzung }} 
\\ \hline 
%\multicolumn{1}{|c|}{\textbf{Time (s)}} &
%\multicolumn{1}{c|}{\textbf{Triple chosen}} 
%\\ \hline 
\endhead


\multicolumn{2}{|l|}%
{{\textbf{Schritt 1 - Starten von RapidForm2004}}}
\\ \hline
Auf dem Desktop doppelt auf das RapidForm Icon klicken.
& 
\includegraphics[width=8cm]{Anleitung/1_Desktop}
\\ \hline 
 
\multicolumn{2}{|l|}%
{{\textbf{Schritt 2 - Oberfläche von RapidForm2004}}}
\\ \hline
Die Oberfläche unterteilt sich in Menü, Werkzeugleisten, Projektbaum und ??.
Je nach dem welche Ansicht in ?? gewählt ist, verändern sich auch das Menü und die Werkzeugleisten. 
& 
\includegraphics[width=8cm]{Anleitung/2_RapidForm}
\\ \hline  

\pagebreak 





\multicolumn{2}{|l|}%
{{\textbf{Schritt 3 - Starten des ''ADD-IN''}}}
\\ \hline
In der Menüzeile auf 
\textbf{Macro -> Addins -> Konica Minolta VIVID Direct Control Addin v2.6.11}
klicken.
\todo{Schrittmotor verbinden!}
& 
\includegraphics[width=8cm]{Anleitung/3_ADDIN_Menu}
\\ \hline  

\multicolumn{2}{|l|}%
{{\textbf{Schritt 4 - Kalibrieren vorbereiten}}}
\\ \hline
\begin{TippS}Für ein erfolgreiches Zusammenführen der einzelnen Aufnahmen ist die Kalibrierung unerlässlich!\end{TippS}
Auf dem Add-In Panel, unter dem Vorschau Fenster, auf \textbf{Live-Preview} klicken. \linebreak
Das Kalibrierungsblech auf dem Drehtisch positionieren.  \linebreak
Dabei muss der Noppen an der Unterseite des Kalibrierungsblechs in das mittlere Loch des Drehtisches gesteckt werden. Die abgeklebte Seite muss zum VI-900 zeigen.
& 
%\includegraphics[width=8cm]{Anleitung/3_ADDIN_Menu}
Bild?
\\ \hline  

\pagebreak 


\multicolumn{2}{|l|}%
{{\textbf{Schritt 5 - Kalibrieren}}}
\\ \hline
Den Reiter \textbf{VIVID: 1} auswählen.\linebreak
Bei \textbf{Manual Para.} ein Häkchen setzen.\linebreak
Im Feld \textbf{Laser Power} ''23'' eintragen.\linebreak
Auf \textbf{Scan for Calib} klicken.
& 
\includegraphics[width=8cm]{Anleitung/4_Calibration}
\\ \hline  

\multicolumn{2}{|l|}%
{{\textbf{Schritt 6 - Kalibrierungsergebnis}}}
\\ \hline
Das Ergebnis sollte ähnlich zu dem in der rechten Abbildung sein. \linebreak
Falls das Add-In einen Fehler ausgibt, muss das Kalibrierungsblech eventuell anders positioniert werden, der Wert im Feld \textbf{Laser Power} verändert werden oder der Fokus manuell eingestellt werden.\linebreak
War die Kalibrierung erfolgreich können die \emph{Kalibrationsebenen }im \emph{Projektbaum} ausgeblendet werden.
& 
\includegraphics[width=8cm]{Anleitung/4_Calibration_result}
\\ \hline  

\pagebreak 


\multicolumn{2}{|l|}%
{{\textbf{Schritt 7 - StepScan}}}
\\ \hline
Bei \textbf{Manual Para.} das Häkchen entfernen.\linebreak
Zum Reiter \textbf{Step} wechseln.\linebreak
Bei \textbf{Angle Tag} und \textbf{Rotate Table to next Scan position} Häkchen setzen.\linebreak
Bei \textbf{Init. Align in RF Using Rotary Info.} und \textbf{Auto Accept} die Häkchen entfernen.\linebreak
Bei \textbf{Rotation Step} die gewünschte Drehung in Grad eingeben. \linebreak
''45'', ''60'' und ''90'' sind gute Werte.
& 
\includegraphics[width=8cm]{Anleitung/5_StepScan.jpg}
\\ \hline  

\multicolumn{2}{|l|}%
{{\textbf{Schritt 8 - AutoFocus}}}
\\ \hline
Das \emph{Kalibrationsblech} entfernen und durch das zu scannende Objekt ersetzen.\linebreak
Zum Reiter \textbf{Control} wechseln. \linebreak
Auf \textbf{Autofocus} klicken.\linebreak
& 
\includegraphics[width=8cm]{Anleitung/6_AutoFocus}
\\ \hline  

\pagebreak 


\multicolumn{2}{|l|}%
{{\textbf{Schritt 9 - Scan}}}
\\ \hline
Auf \textbf{Scan} klicken. \linebreak
Das Objekt sollte möglichst schon zu erkennen sein und die Farben sich im Mittleren Bereich bewegen. \linebreak
Ansonsten muss mit den Parametern \textbf{Focus} und \textbf{LaserPower gespielt werden.}
\begin{TippS}Die Position des VI-900 darf nach der Kalibrierung nicht mehr verändert werden!\end{TippS}
& 
\includegraphics[width=8cm]{Anleitung/7_Scan}
\\ \hline  

\multicolumn{2}{|l|}%
{{\textbf{Schritt 10 - Akzeptieren}}}
\\ \hline
Wenn das Objekt gut zu erkennen ist, werden mit \textbf{Accept} die Daten an RapidForm2004 gesendet. Der Drehtisch sollte sich nun automatisch um den eingestellten Winkel drehen. 
\begin{TippS}Bei \textbf{AutoAccept} kann nun ein Häkchen gesetzt werden.\end{TippS}
& 
\includegraphics[width=8cm]{Anleitung/8_Accept}
\\ \hline  

\pagebreak 


\multicolumn{2}{|l|}%
{{\textbf{Schritt 11 - Scannen}}}
\\ \hline
Auf \textbf{Scan} klicken.\linebreak
Diesen Schritt wiederholen bis alle Aufnahmen abgeschlossen sind.\linebreak
& 
\includegraphics[width=8cm]{Anleitung/8_AutoAccept}
\\ \hline  

\multicolumn{2}{|l|}%
{{\textbf{Schritt 12 - Ergebnis der Scans}}}
\\ \hline
Nach Abschluss aller Scans dreht der Tisch sich automatisch in die Ursprungsposition zurück.\linebreak
Im Arbeitsbereich sollten sich nun alle Scans befinden.
& 
\includegraphics[width=8cm]{Anleitung/9_1_Scan_result}
\\ \hline  

\pagebreak 


\multicolumn{2}{|l|}%
{{\textbf{Schritt 13 - Drehen und Zusammenführen der Scans}}}
\\ \hline
In der Menüzeile auf \textbf{Build -> Register -> Rotary Table} klicken.\linebreak
& 
\includegraphics[width=8cm]{Anleitung/10_Rotary_Table}
\\ \hline  

\multicolumn{2}{|l|}%
{{\textbf{Schritt 14 - Registrieren}}}
\\ \hline
Im Dialog auf \textbf{Select Axis} klicken.\linebreak
Im Darstellungsbereich auf die Achse aus dem Kalibrationsscan klicken.
Bei \textbf{Rotation-Angle} den Winkel eines Scans eintragen.\linebreak
Im \emph{Projektbaum} den Entsprechenden Scan auswählen.\linebreak
Die letzten beiden Schritte mit allen Scans wiederholen.\linebreak
Den Dialog mit \textbf{Ok} verlassen.
& 
\includegraphics[width=8cm]{Anleitung/11_Register_Dialog_2}
\\ \hline  

\pagebreak 
 

\multicolumn{2}{|l|}%
{{\textbf{Schritt 15 - Ergebnis}}}
\\ \hline
Als Ergebnis sollte nun ein komplettes 3D-Modell des Objektes herauskommen.
& 
\includegraphics[width=8cm]{Anleitung/11_Register_result}
\\ \hline  




\end{longtable} 

