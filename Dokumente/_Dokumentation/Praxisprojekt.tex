% anstelle von article auch KOMA-Skript-Variante scrartcl
\documentclass[11pt]{article}

%anstelle von german auch ngerman
\usepackage{amsfonts,latexsym,eurofont,german}
\usepackage{graphicx,color}
%titelblatt

\thispagestyle{empty}

\setlength{\textheight} {270mm} 
\setlength{\textwidth} {150mm}

%Kopfzeile
\setlength{\topmargin} {-25mm}

%Fu�zeile
%\setlength{\footheight} {10mm}
\setlength{\footskip} {10mm}


%Seiten link
\setlength{\oddsidemargin} {10mm} 

%Seite rechts und Rand
\setlength{\marginparwidth} {0mm} 
\setlength{\marginparsep} {0mm} 

%Abs�tze nicht einr�cken, m�glicherweise nicht zu empfehlen, falls viele Abbildungen vorhanden;
%Betreuer fragen
\parindent0mm

%Inhaltsverzeichnis
\usepackage[numindex,nottoc,section]{tocbibind} %Einbinden von Index-
                                %und Literaturverzeichnis (nummeriert) in das
                                %Inhaltsverzeichnis. Inhaltsverzeichnis wird 
                                %nicht eingebunden.


\begin{document}


\begin{titlepage}
\author{}
%Bilder einf�gen
\begin{figure}[hp]
	\centering
	\includegraphics{bsp.jpg}
	\hfill
	\includegraphics[height = 13mm]{fh-remagen.jpg}
\end{figure}
\begin{center}
\huge
\textsc{Titel\\ des Praxisprojektes}\\

\vspace{2cm}

\LARGE
%\textsc{\bf Bachelorarbeit\\[0.5\baselineskip]
\textsc{\bf Praxisprojektbericht\\[0.5\baselineskip]
\Large \normalfont {im Studiengang XXXXXXXX\\[0.5\baselineskip]
Fachhochschule Koblenz, RheinAhrCampus Remagen}}

\large
\vspace{2cm}
\textnormal{vorgelegt von\\[0.5\baselineskip]\textbf{Max Mustermann}\\[0.5\baselineskip] geb.\ am 01.01.1991 in {\em Musterstadt} }\\ 
\vspace{1cm}


\end{center}

\normalsize
\begin{flushleft}	

\vspace{15mm}
\begin{tabbing}
Externer Betreuer:\quad\=\kill
Interner Betreuer: \> Prof.\ Dr.\ A\\
Externer Betreuer: \> Dr.\ C\\[1\baselineskip]

durchgef\"uhrt bei: \> Firma XYZ\\
\end{tabbing}
\end{flushleft}

%\vspace{0mm} 
% falls vertraulich
\begin{center}
VERTRAULICH
\end{center}

\vspace{1cm}

\begin{center}
\textnormal{Remagen, \today}
\end{center}


\end{titlepage}

% Erstellung des Inhaltsverzeichnisses
\tableofcontents \newpage

% %Erstellung des Abbildungsverzeichnis
 \listoffigures \newpage
 
% Erstellung des Tabellenverzeichnisses
\listoftables\newpage


\newpage

\section{Einleitung}
Die Aufgabenstellung geht auf eine Kooperation zwischen
\begin{figure}[hp]
	\centering
	\includegraphics[height = 13mm]{fh-remagen.jpg}
	\caption{FH-Logo}
\end{figure}

und 

\begin{figure}[hp]
	\centering
	\includegraphics[height = 13mm]{bsp.jpg}
	\caption{Firma}
\end{figure}

im Bereich der Bewegungsanalyse zur\"uck.

\newpage

\section{Das Unternehmen}
Vorstellung des Unternehmens und der Abteilung, in der das Praxisprojekt durchgef\"uhrt wurde.

\section{Mathematische Grundlagen}

TEXT

\subsection{Der Chi--Quadrat-Test}
TEXT


\subsection{Risikoma\3e}

\newpage

\section{AMA -- Ansatz}

\newpage

\section{Zusammenfassung}

Eine abschlie\3ende Analyse f\"uhrt zu 

\newpage

\addcontentsline{toc}{section}{A Anhang 1}
\section*{A Anhang 1}


\newpage

\begin{thebibliography}{99}
%Zeitschriftenartikel
\bibitem{maquarbra} Mack, T., Quarg, G., Braun, C.\ (2006). {\em The mean square error of prediction in the chain ladder reserving method. A comment.} ASTIN Bulletin 36, 543-553. 
%Buch
\bibitem{mik} Mikosch, T.\ (1994). {\em Non-life insurance mathematics.} Springer, Heidelberg.
%Internetquelle
\bibitem{wiki1} Wikipedia. {\em Chi--Quadrat--Test}, http://de.wikipedia.org/wiki/Chi-Quadrat-Test, Stand: 30.09.2011.
\end{thebibliography}

\newpage

\section*{Erkl\"arung}

\vspace{2cm}

Hiermit versichere ich, dass ich den vorliegenden Bericht selbst\"andig und nur unter Verwendung der angegebenen Quellen und Hilfsmittel verfasst habe.

\vspace{2cm}

Remagen, den \today \hfill {Max Mustermann} 



\end{document}
