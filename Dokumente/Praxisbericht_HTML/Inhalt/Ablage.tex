
\subsection{Vorgaben}
Im konkreten Beispiel ging es um die automatische 3D-Erfassung eines Schädelmodells. In der Ausgangssituation war es zwar möglich, mit der Erfassungssoftware RapidForm2004 und dem 3D-Laserscanner VI-900  einen Schädel aus einer Richtung zu erfassen. Die Kommunikation zwischen dem VI-900 und der Erfassungssoftware funktionierte. Die Drehtischsteuerung, welche den Drehtisch nach den Vorgaben der Erfassungssoftware des Computers drehen sollte, war nicht in das System eingebunden. Dies war ein Problem der verschiedenen Befehlssätze.\\
Aufbau eines Übersetzers, basierend auf einem Mikrocontroller. Der Übersetzer sollte ein LC-Display, mehrere Taster, mehrere LEDs und zwei serielle Schnittstellen enthalten. Die Höhenverstellung des Drehtisches sollte genutzt werden und die zu Beginn noch nicht funktionierenden Endschalter sollten die vorgesehene Funktion erfüllen.
\subsection{Zielvorgabe}
\begin{enumerate}
\item Der VI-900 erstellt eine Aufnahme des Schädels.
\item Der VI-900 sendet die Aufnahme an die Erfassungssoftware im Computer.
\item Die Erfassungssoftware im Computer sendet nach der Speicherung der Aufnahme den Befehl zum Drehen des Drehtisches an die Drehtischsteuerung.
\item Die Drehtischsteuerung dreht den Tisch um die gewünschte Gradzahl.
\item Die Drehtischsteuerung meldet die erfolgreiche Rotation mit Hilfe des zu entwickelnden Übersetzers an die Erfassungssoftware im Computer.
\item Die Erfassungssoftware im Computer sendet erneut einen Aufnahmebefehl an den VI-900.
\end{enumerate}


Im konkreten Fall soll nun die Erstellung von 3D-Daten eines vorhanden Objektes (\Fachbegriff{Reverse Engineering}) genutzt werden.\\
Es wird nun mit einer Kombination aus dem 3D-Laserscanner, dem Drehtisch und der dazugehörigen Erfassungssoftware ein 3D-Modell erfasst. Dieses kann dann unterstützend in der \Fachbegriff{CAD-Entwicklung} genutzt werden kann.\\
In der CAD-Entwicklung kann es vorkommen das für ein real existierendes Objekt eine \textit{Erweiterung} konstruiert werden soll. Um die Erweiterung, einen Anschlag zum Beispiel, leichter konstruieren zu können, ist es von Vorteil, die Abmessungen des Objektes möglichst genau zu kennen. Das Übertragen der Abmessungen in die CAD-Software kann, insbesondere für komplexe Objekte, sehr aufwendig sein.
Abhilfe soll der 3D-Laserscanner schaffen, der das Objekt aus mehreren Richtungen vermisst und aus diesen Informationen ein 3D-Modell generiert. Dieses soll dann in einem neutralen CAD-Format \todo{welches?} exportiert werden.









Dazu sollen die Befehle der Software mit einem Mikrocontroller ausgewertet werden und in, für den Drehtisch, verständliche Befehle übersetzt.
Für die Höhenverstellung des Drehtisches wird eine manuelle Steuerung im Mikrocontroller realisiert und die Endschalter so verdrahtet das sie funktionieren. \\
Der Mikrocontroller lässt sich mit mehreren Tastern bedienen und ein \Fachbegriff{LC-Display} zeigt den aktuellen Status an.\\
Der Mikrocontroller und seine Peripherie werden als \Fachbegriff{19''-Einschub} realisiert, da die Ansteuerung des Drehtisches auch als Einschub realisiert ist.\\

Der Aufbau der Arbeit gliedert sich im Wesentlichen in die Nutzung vorhandener und die Entwicklung neuer Hardware, sowie in die Entwicklung der Software für den Mikrocontroller und eine Schritt-für-Schritt Anleitung. \\
Zur Hardware gehören der 3D-Laserscanner, die Ansteuerung für den Drehtisch sowie der Drehtisch selbst, seine Spannungsversorgung, die Schrittmotoren und die Schrittmotorkarten, die Motorverkabelung, die Endschalter, sowie der Mikrocontroller, der Pegelwandler MAX232, ein LC-Display, als auch das Platinenlayout und der 19''-Einschub.\\
Zur Software gehören die 3D-Erfassungsoftware, die Entwicklungsumgebungen und die Software für den Mikrocontroller. Die Software für den Mikrocontroller deckt das Reverse-Engineering der Protokolle, deren Auswertung und Übersetzung ab; außerdem eine manuelle Ansteuerung des Drehtisches.\\
Im Anhang befindet sich eine Schritt-für-Schritt Anleitung, 3D-Modelle aufzunehmen und zu exportieren.


\begin{figure}[htb]
\centering
\includegraphics[width=\textwidth]{STK_500}
\caption{STK500}
\label{fig:STK500}
\end{figure}
\begin{figure}[htb]
\centering
\includegraphics[width=\textwidth]{STK500_Schema}
\caption{STK500 - Schema}
\label{fig:STK500_Schema}
\end{figure}